\documentclass{scrartcl}
\usepackage{style}

\newcommand{\versionmajor}{0}
\newcommand{\versionminor}{1}
\newcommand{\versionpatch}{0}
\newcommand{\version}{\versionmajor.\versionminor.\versionpatch}

\title{\LARGE
    Kotlin DSL for BDI Agents development
}

\subtitle{Project Proposal}

\author{
    Baiardi Martina \\ \emailaddr{martina.baiardi4@studio.unibo.it}
}

\date{\today}

\begin{document}

\maketitle

\begin{abstract}
    The aim of the project is to create a domain specific language to develop agents following the BDI (Beliefs, Desires, Intetions) model defined by Rao and Georgeff \cite{rao1995bdi}.
    The library provides a tool to develop agents who perceive and act within a shared environment and communicate with each other.
\end{abstract}

\section{Goals/requirements}

The goal that the project wants to achieve is the development of a framework to define BDI agents.
The fundamental entities that will be modeled are \textbf{Agent} and \textbf{Environment}.

\smallskip

Agents coexist within the environment and communicate with each other through a form of message passing communication.
Following the Rao and Georgeff's model definition \cite{rao1995bdi}, BDI agents are composed by: \textbf{Beliefs}, \textbf{Desires} and \textbf{Intentions}.

\bigskip

The system needs information about the environment, but the latter changes its state in a nondeterministic way over time
and a single agent's sensing action cannot capture the state correctly.
\textbf{Beliefs} are the agents' component that collect information about the environment state and are updated appropriately after each sensing.

\bigskip

\textbf{Desires} collect agent's objectives that need to be accomplished.
Each desire has its own priority and payoff.

\bigskip

\textbf{Intentions} are the agent's component that collects the currently chosen courses of action to reach the system's objectives.
Actions are appropriately chosen by a \textbf{selection function} that takes the system's beliefs and desires as input.


\subsection{Scenarios}

The following project wants to provide a clear framework to build a Multi-Agent System (MAS) using a domain specific language.
Users will be able to:
\begin{itemize}
    \item Create an environment shared among multiple agents.
    \item Define environment's properties that agents can perceive and modify.
    \item Develop BDI agents and define their beliefs and desires.
    \item Build an agent system where agents can communicate with each other.
\end{itemize}

%\subsection{Self-assessment policy}
%
%\begin{itemize}
%    \item How should the \emph{quality} of the \emph{produced software} be assessed?
%
%    \item How should the \emph{effectiveness} of the project outcomes be assessed?
%\end{itemize}

%\section{Background}
%
%Any preliminary notion necessary to understand the goals, the motivations, the context, or the development of this project.
%%
%\begin{itemize}
%    \item Theoretical aspects
%
%    \item Used frameworks / models / technologies (and motivation for their choice)
%\end{itemize}
%
%\section{Requirements Analysis}
%
%Is there any implicit requirement hidden within this project's requirements?
%%
%Is there any implicit hypothesis hidden within this project's requirements?
%%
%Are there any non-functional requirements implied by this project's requirements?
%
%What model / paradigm / techonology is the best suited to face this project's requirements?
%%
%What's the abstraction gap among the available models / paradigms / techonologies and the problem to be solved?
%
%\section{Design}
%
%This is where the logical / abstract contribution of the project is presented.
%
%Notice that, when describing a software project, three dimensions need to be taken into account: structure, behaviour, and interaction.
%
%Always remember to report \textbf{why} a particular design has been chosen.
%Reporting wrong design choices which has been evalued during the design phase is welcome too.
%
%\subsection{Structure / Domain Entities}
%
%Which entities need to by modelled to reflect the domain?
%%
%(UML Class diagram)
%
%How should entities be modularised?
%%
%(UML Component / Package / Deployment Diagrams)
%
%\subsection{Behaviour}
%
%How should each entity behave?
%%
%(UML State diagram or Activity Diagram)
%
%\subsection{Interaction}
%
%How should entities interact with each other?
%%
%(UML Sequence Diagram)
%
%\section{Implementation Details}
%
%Just report interesting / non-trivial / non-obvious implementation details.
%
%This section is expected to be short in case some documentation (e.g. Javadoc or Swagger Spec) has been produced for the software artefacts.
%%
%This this case, the produced documentation should be referenced here.
%
%\section{Self-assessment / Validation}
%
%Choose a criterion for the evaluation of the produced software and \textbf{its compliance to the requirements above}.
%
%Pseudo-formal or formal criteria are preferred.
%
%In case of a test-driven development, describe tests here and possibly report the amount of passing tests, the total amount of tests and, possibly, the test coverage.
%
%\section{Deployment Instructions}
%
%Explain here how to install and launch the produced software artefacts.
%%
%Assume the softaware must be installed on a totally virgin environment.
%%
%So, report \textbf{any} configuration step.
%
%Gradle and Docker may be useful here to ensure the deployment and launch processes to be easy.
%
%\section{Usage Examples}
%
%Show how to use the produced software artefacts.
%
%Ideally, there should be at least one example for each scenario proposed above.
%
%\section{Conclusions}
%
%Recap what you did
%
%\subsection{Future Works}
%
%Recap what you did \emph{not}
%
%\subsection{What did we learned}
%
%Racap what did you learned
%
%\section*{Stylistic Notes}
%
%Use a uniform style, especially when writing formal stuff: $X$, X, $\mathbf{X}$, $\mathcal{X}$, \texttt{X} are all different symbols possibly referring to different entities.
%
%This is a very short paragraph.
%
%This is a longer paragraph (notice the blank line in the code).
%It composed by several sentences.
%%
%You're invited to use comments within \texttt{.tex} source files to separate sentences composing the same paragraph.
%
%Paragraph should be logically atomic: a subordinate sentence from one paragraph should always refer to another sentence from within the same paragraph.
%
%The first line of a paragraph is usually indented.
%%
%This is intended: it is the way \LaTeX{} lets the reader know a new paragraph is beginning.
%
%\begin{figure} % DO NOT write any positional hint (e.g. [h] or [t] here!
%    \centering
%    \includegraphics[width=0.5\linewidth]{figures/universe.jpg}
%    \caption{Some floating image}
%    \label{fig:image}
%\end{figure}
%
%Let \LaTeX{} decide where to put figures (or tables, or listings), label them and reference the labels instead of say things like ``in the following image...''.
%%
%Consider for instance the case of \cref{fig:image}.
%
%Use the \href{https://en.wikibooks.org/wiki/LaTeX/Source_Code_Listings}{\texttt{listing}} package for inserting scripts into the \LaTeX{} source.
%%
%Consider for instance \cref{lst:snippet}.
%
%% have a look to macros in code-listings.sty
%\javaimport[
%    caption={Some Java listing},
%    label={lst:snippet}
%]{listings/HelloWorld.java}

\nocite{*} % Includes all references from the `references.bib` file
\bibliographystyle{plain}
\bibliography{references}

\end{document}
